\chapter{Introducción}
	
	El caos se refiere a un tipo de comportamiento dinámico complejo que posee algunas características muy especiales, tales como extrema sensibilidad a pequeñas variaciones de la condición inicial, trayectorias encerradas en el espacio de fase pero con un exponente de Lyapunov positivo, un espectro de potencia continuo entre muchas otras. En pocas palabras, el caos es simplemente un comportamiento impredecible de un sistema determinista. Es de interés resaltar que los sistemas caóticos ya eran conocidos desde hace mucho tiempo atrás y que no fue hasta hace poco que se logró demostrar que el caos puede ser controlado y debido a esto impactar en muchas áreas, tanto en áreas cercanas a la electrónica como técnicas de modulación, sistemas de comunicación, técnicas de encriptación de datos, como también en áreas  relacionadas a los sistemas biológicos, reacciones químicas, toma de decisiones críticas en política, economía, eventos militares, etc  \cite{Munoz-Pacheco2010}. El caos es un fenómeno que ocurre en muchos sistemas no lineales, donde la naturaleza determinista de la estructura se conjuga con la irregularidad del comportamiento, esto significa que, a pesar del hecho de que el sistema se describe mediante un conjunto de ecuaciones diferenciales ordinarias, donde todos los términos son perfectamente conocidos, su comportamiento es irregular y muy sensible a las condiciones iniciales. La primera evidencia de imprevisibilidad en los sistemas deterministas se encuentra en el trabajo del matemático y científico Henri Poincaré sobre el movimiento celestial, mientras que la primera formulación del caos en un modelo matemático expresado por un conjunto de ecuaciones diferenciales ordinarias que exhiben el caos se debe al matemático y meteorólogo Edward Lorenz que estaba estudiando un modelo de movimiento del aire en la atmósfera y descubrió cómo pequeñas variaciones en los valores iniciales de las variables de su modelo dieron como resultado predicciones meteorológicas divergentes \cite{Buscarino2014}. Para el momento de estos estudios faltaba una prueba experimental definitiva del caos y la tecnología y poder de cómputo no eran suficientes para pensar aún en soluciones y ni pensar en aplicaciones. Debido al constante avance de la electrónica, hoy en día somos capaces de sintetizar mediante dispositivos electrónicos sistemas caóticos, utilizando técnicas de modelado e implementación es posible crear representaciones de estos, no obstante, todas estas se basan en aproximaciones que aún no han sido exploradas en su totalidad. 
	Por otro lado el cálculo fraccionario es un tema que tiene más de 300 años de antigüedad y que se remonta a cartas enviadas a Leibnitz por parte de Bernoulli y de L'Hôspital preguntando acerca de la derivada a la $1/2$ e indagando sobre su significado. Con el paso de los años la teoría de cálculo fraccionario se fue desarrollando pasando por las manos de nombres conocidos como Euler, Lagrange, Laplace, Fourier hasta llegar a Liouville, Riemann, Grünwald, Letnikov, Caputo entre muchos otros \cite{Petras2011}. Pero aún con todo ese desarrollo no fue hasta hace poco que la comunidad científica comenzó a interesarse por esta rama del cálculo y la razón principal de este cambio es que los cálculos necesarios para cualquier posible implemetación eran demasiado complejos y lentos, un panorama totalmente diferente se vive en la actualidad, el rápido avance de la tecnología ha logrado realizar avances notables en esta área. El rango de aplicaciones para el cálculo fraccionario es inmenso, por mencionar algunas de las recientes en los últimos años, la modelación de derivadas fraccionarias para obtener una mejor representación comportamental de un sistema industrial metalúrgico \cite{Petras2019}, la incorporación de dinámica de orden fraccional para mejorar la robustez de un control PI/PID para motores DC \cite{Tepljakov2016,Khubalkar2018}, la modelación de señales biológicas como ECG, EMG y EEG debido a su respuesta de magnitud de 20$\alpha$dB, modelos fisiológicos basados en ecuaciones diferenciales lineales que describen fenómenos complejos en el cuerpo humano como la oxigenación de la sangre entre otros \cite{Ortigueira2011}.
	El cálculo fraccionario y los sistemas caóticos se complementan al añadir un nivel de profundidad en la creación de osciladores caóticos modelados como un conjunto de ecuaciones diferenciales no lineales fraccionarias las cuales son materia prima para la creación de nuevas aplicaciones y áreas de desarrollo. 
	
	\section{Justificación}
	
	Los osciladores caóticos son una área de oportunidad emergente cuyas aplicaciones han aumentado en los ultimamos años
	\section{Objetivos}
	
		\subsection{Objetivo general}
		
		\subsection{Objetivos específicos}
	