%-------------------------------------------------------------------------------
%                                Paquetes extras                               %
%-------------------------------------------------------------------------------
\usepackage{lipsum}												% Texto de ejemplo \lipsum[1-30]
\decimalpoint													% Punto decimal en lugar de coma
\spanishsignitems												% Viñetas en lugar de cuadros
\raggedbottom													% Eliminar molestos warnings

\usepackage{pdfpages}											% Incluir portada echa en Inkscape
\usepackage{setspace}											% Interlineado
\usepackage{makecell}											% Para tablas

\usepackage[nottoc]{tocbibind}									% Bibliografica en table of contents

\usepackage[figuresright]{rotating}								% Rotar figuras con caption
%-------------------------------------------------------------------------------
%                            Comandos matematicos                              %
%-------------------------------------------------------------------------------
\usepackage{steinmetz}											% Para representar fasores
\usepackage{bm}													% Bold math  \bm command
\newcommand{\binomb}[2]{\genfrac{[}{]}{0pt}{}{#1}{#2}}
%-------------------------------------------------------------------------------
%                        Paquetes para hipervinculos                           %
%-------------------------------------------------------------------------------
\usepackage[hidelinks]{hyperref}								% Añade los bookmarks y le quita la caja roja, \url{}
%\urlstyle{same}
%-------------------------------------------------------------------------------
%                           Estilos de encabezados                             %
%-------------------------------------------------------------------------------
\usepackage{fancyhdr, blindtext}								% Libreria para encabezados

\renewcommand{\chaptermark}[1]{\markboth{#1}{}}					% Capitulos y secciones en minusculas
\renewcommand{\sectionmark}[1]{\markright{#1}}

\fancypagestyle{normalstyle}{%
  \fancyhf{}													% Reinicial estilos de header y footer
	\fancyhead[LE,RO]{\thepage}
	\fancyhead[LO]{\nouppercase{\rightmark}}
	\fancyhead[RE]{\nouppercase{\leftmark}}
	\renewcommand{\headrulewidth}{0.4pt}
	\renewcommand{\footrulewidth}{0pt}
	\setlength{\headheight}{14.62pt}
}
%-------------------------------------------------------------------------------
%                            Libreria de codigos                               %
%-------------------------------------------------------------------------------
% Paquetes necesarios
\usepackage{listings}
\usepackage{xcolor}

% Tipos de letra personalizadas
\def\lstbasicfont{\fontfamily{pcr}\selectfont\scriptsize}
\def\vhdlbasicfont{\fontfamily{cmtt}\selectfont\scriptsize}

% Colores personalizados
\definecolor{codegreen}{rgb}{0,0.6,0}
\definecolor{codepurple}{rgb}{0.58,0,0.82}

\definecolor{codegray}{rgb}{0.5,0.5,0.5}
\definecolor{backcolour}{rgb}{0.95,0.95,0.92}
\definecolor{codeorange}{RGB}{254, 100, 35}

% Deficion de lenguajes perzonalizados

% Definicion de lenguaje MATLAB
\lstdefinelanguage{matlabfloz}{%
  alsoletter={...},%
  morekeywords={%                             % keywords
		break,case,catch,classdef,continue,else,
		elseif,end,for,function,global,if,
		otherwise,parfor,persistent,
		return,spmd,switch,try,while,...},        % Use the matlab "iskeyword" command to get those
  comment=[l]\%,                              % comments
  morecomment=[l]...,                         % comments
  morecomment=[s]{\%\{}{\%\}},                % block comments
  morestring=[m]'                             % strings 
}[keywords,comments,strings]%

% Estilos MATLAB
\lstdefinestyle{MATLAB}{
	frame=single,
	rulecolor=\color{black},
	framexleftmargin=4mm,
	xleftmargin=2mm,
	language=matlabfloz,
  commentstyle=\color{codegreen},
  keywordstyle=\color{blue}, %magenta
  numberstyle=\tiny\color{black},
  stringstyle=\color{codepurple},
  basicstyle=\lstbasicfont\scriptsize,
  breakatwhitespace=false,         
  breaklines=true,                 
  captionpos=b,                    
  keepspaces=true,                 
  numbers=left,                    
  numbersep=5pt,                  
  showspaces=false,                
  showstringspaces=false,
  showtabs=false,                  
  tabsize=2    
}

% Estilos MATLAB en codigo
\lstdefinestyle{MATLAB_preview}{
%	frame=single,
%	rulecolor=\color{black},
%	framexleftmargin=4mm,
%	xleftmargin=2mm,
	language=matlabfloz,
    commentstyle=\color{codegreen},
    keywordstyle=\color{blue}, %magenta
%    numberstyle=\tiny\color{black},
    stringstyle=\color{codepurple},
    basicstyle=\lstbasicfont\small,
    breakatwhitespace=false,         
    breaklines=true,                 
    captionpos=b,                    
    keepspaces=true,                 
%    numbers=left,                    
%    numbersep=5pt,                  
    showspaces=false,                
    showstringspaces=false,
    showtabs=false,                  
    tabsize=2    
}

\renewcommand{\lstlistingname}{Código}% Listing -> Algorithm
\renewcommand{\lstlistlistingname}{Lista de códigos}% 
%-------------------------------------------------------------------------------
%                           Caption en negritas                                %
%-------------------------------------------------------------------------------
\usepackage[labelfont=bf]{caption}
\captionsetup{labelfont=bf}